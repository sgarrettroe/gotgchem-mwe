%%% R1 Identify information about an atom (any isotope or ion) related to the number of protons, neutrons, electrons, mass number, atomic number, and the overall charge

\titledquestion{RCLK7981} %%%SLO:eylsza

\begin{pycode}

def AtomicInfo1():
	g = pt.elements[randrange(4,37)]
	maxabundance = 0
	for i in range(len(g.isotopes)):
		if g[g.isotopes[i]].abundance > maxabundance:
			iso = i
			maxabundance = g[g.isotopes[i]].abundance
	massnumber = g.isotopes[iso]
	protons = g.number
	neutrons = g.isotopes[iso]-protons
	charge = 0
	while charge == 0:
		charge = randrange(-3,4)
	electrons = protons - charge
	if charge > 0:
		charge = str(charge)+'+'
	else: 
		charge = str(charge)
	thereturnstring = '\\renewcommand{\\massnumber}{'+str(massnumber)+'} \
	\\renewcommand{\\protons}{'+str(protons)+'} \
	\\renewcommand{\\neutrons}{'+str(neutrons)+'} \
	\\renewcommand{\\electrons}{'+str(electrons)+'} \
	\\renewcommand{\\name}{'+g.name+'} \
	\\renewcommand{\\charge}{'+charge+'}'
	return thereturnstring

\end{pycode}

\providecommand{\massnumber}{init}
\providecommand{\protons}{init}
\providecommand{\neutrons}{init}
\providecommand{\electrons}{init} 
\providecommand{\charge}{init}
\providecommand{\name}{init}

\py{AtomicInfo1()}

The most common isotope of \name{} has \neutrons{} neutrons. For the \charge{} ion of this isotope, identify each of the following: 

\begin{parts}

  \part the atomic number: \fillin[\protons]
   
  \part the mass number: \fillin[\massnumber]
  
  \part the number of electrons: \fillin[\electrons]

\end{parts}

\vspace{\stretch{1}}

%%%%%%%%%%%%%%%%%%%%%%%%%%%%%%%%%%%%%%

\titledquestion{YCQC4381} %%%SLO:eylsza

\begin{pycode}

def AtomicInfo3():
	g = pt.elements[randrange(4,37)]
	maxabundance = 0
	for i in range(len(g.isotopes)):
		if g[g.isotopes[i]].abundance > maxabundance:
			iso = i
			maxabundance = g[g.isotopes[i]].abundance
	massnumber = g.isotopes[iso]
	protons = g.number
	neutrons = g.isotopes[iso]-protons
	charge = randrange(-3,4)
	electrons = protons - charge
	while charge == 0:
		charge = randrange(-3,4)
	electrons = protons - charge
	if charge > 0:
		charge = '+'+str(charge)
	elif charge < 0:  
		charge = str(charge)
		
	thereturnstring = '\\renewcommand{\\massnumber}{'+str(massnumber)+'} \
	\\renewcommand{\\protons}{'+str(protons)+'} \
	\\renewcommand{\\neutrons}{'+str(neutrons)+'} \
	\\renewcommand{\\electrons}{'+str(electrons)+'} \
	\\renewcommand{\\name}{'+g.name+'} \
	\\renewcommand{\\mycharge}{'+charge+'}'
	return thereturnstring

\end{pycode}

\providecommand{\massnumber}{init}
\providecommand{\protons}{init}
\providecommand{\neutrons}{init}
\providecommand{\electrons}{init} 
\providecommand{\mycharge}{init}
\providecommand{\name}{init}

\py{AtomicInfo3()}

Identify the following for an isotope of \name{} with a mass number of \massnumber{} and \electrons{} electrons. 
 

\begin{parts}

  \part the number of protons: \fillin[\protons]
   
  \part the number of neutrons: \fillin[\neutrons]
  
  \part the charge: \fillin[\mycharge]

\end{parts}

\begin{solution}
  \ifgradingnotes
 Grading Note: Minor error only if clear math error (e.g., $52 - 24 = 22$). The sign of the charge must be correct.  
\fi
\end{solution}


\vspace{\stretch{0}}


%%%%%% R2 Convert between number of atoms, grams, and moles of an element


\titledquestion{SITM4673} %%%SLO:ejscop

\begin{pycode}

def AtomMass():
	atom = pt.elements[randrange(1,37)]
	num = uniform(1, 10)
	soln = num*1E21/N_A*atom.mass
	thereturnstring = '\\renewcommand{\\theanswer}{\\num[round-mode=figures,round-precision=3]{'+str(soln)+'}}  \
	\\renewcommand{\\atom}{'+atom.name+'} \
		\\renewcommand{\\atomicmass}{\\num[round-mode=figures,round-precision=4]{'+str(atom.mass)+'}}\
	\\renewcommand{\\mynum}{\\num[round-mode=figures,round-precision=3]{'+str(num)+'}}'
	return thereturnstring

\end{pycode}

\providecommand{\theanswer}{init}
\providecommand{\atom}{init}
\providecommand{\atomicmass}{init}
\providecommand{\mynum}{init} 

\py{AtomMass()}

What is the mass in grams of \mynum$\times 10^{21}$ \atom{} atoms? \answerline[\theanswer{} g]

\begin{solution}
\[
  \frac{\mynum\times 10^{21} \,\cancel{\text{atoms}}}{1}\,
   \frac{1\,\cancel{\text{mol}}}{6.02\times 10^{23}\,\cancel{\text{atoms}}}\,
    \frac{\atomicmass\,\text{g}}{1\,\cancel{\text{mol}}}=\theanswer\,\text{g}
\]

\vspace{2\baselineskip}
  \ifgradingnotes
 Grading Note: Answer is proficient, if the work shown is correct. 
\fi
\end{solution}

\vspace{\stretch{0.5}}


%% R3 Calculate the average atomic mass of an element given its isotopes and their natural abundance

\titledquestion{RQPY1410} %%%SLO:qyechg

\begin{pycode}

def AvgMass():
	isolist = [3, 5, 29, 31, 35, 37, 47, 49, 51, 63, 71, 75, 77]
	nel = randrange(len(isolist))
	g = pt.elements[isolist[nel]]
	k = 0
	mass = []
	abundance =[]
	avg = 0
	for i in range(len(g.isotopes)):
		if g[g.isotopes[i]].abundance > 0:
			mass.append(g[g.isotopes[i]].mass)
			abundance.append(g[g.isotopes[i]].abundance)
			avg = avg + g[g.isotopes[i]].mass*g[g.isotopes[i]].abundance/100
	thereturnstring = '\\renewcommand{\\theanswer}{\\num[round-mode=figures,round-precision=4]{'+str(avg)+'}}  \
	\\renewcommand{\\massone}{\\num[round-mode=figures,round-precision=4]{'+str(mass[0])+'}}  \
	\\renewcommand{\\masstwo}{\\num[round-mode=figures,round-precision=4]{'+str(mass[1])+'}}  \
	\\renewcommand{\\abundanceone}{\\num[round-mode=figures,round-precision=3]{'+str(abundance[0])+'}}  \
		\\renewcommand{\\abundancetwo}{\\num[round-mode=figures,round-precision=3]{'+str(100-abundance[0])+'}}  \
	\\renewcommand{\\name}{'+g.name+'}' 
	return thereturnstring

\end{pycode}

\providecommand{\theanswer}{init}
\providecommand{\massone}{init}
\providecommand{\masstwo}{init}
\providecommand{\abundanceone}{init}
\providecommand{\abundancetwo}{init}
\providecommand{\name}{init}


\py{AvgMass()}

An element has two naturally occurring isotopes. Isotope 1 has a mass of \massone{} amu and a relative abundance of \abundanceone{}\%, and isotope 2 has a mass of \masstwo{} amu. Calculate the average atomic mass of this element. \answerline[\theanswer{} amu, \name]

\begin{solution}
\[
(\massone\,\text{amu})\left(\frac{\abundanceone}{100}\right)+(\masstwo\,\text{amu})\left(\frac{\abundancetwo}{100}\right)=\theanswer\,\text{amu, \name}
\]

\vspace{2\baselineskip}
  \ifgradingnotes
 Grading Note: Answer is proficient, if the work shown is correct and the two percentages add up to 100\%. 
 \fi
\end{solution}

\vspace{\stretch{0.5}}

