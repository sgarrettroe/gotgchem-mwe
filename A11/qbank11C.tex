% Copyright (C) 2023 Tricia D. Shepherd, Sean Garrett-Roe

%%% C1 Write or interpret nuclide symbols to represent an element, ion, or isotope

\titledquestion{XOIQ4653} %%%SLO:xzenop

\begin{pycode}

template_string = r'''

Write the nuclide symbols of two isotopes of a @num ion with @electrons electrons.

\begin{solution}
\ce{^x@ion} and \ce{^y@ion} - where $x\neq y$ and both numbers are greater than @protons.

\vspace{2\baselineskip}
\end{solution}
'''

g = pt.elements[randrange(4,37)]
protons = g.number
charge = 0
while charge == 0:
	charge = randrange(-3,4)
electrons = protons - charge
if charge > 0:
	charge = eatOnes(charge)
	charge = '+'+charge
elif charge < 0: 
	charge = eatOnes(charge)
	num = charge
	
if charge == '+':
	num = '+1'
else:
	num = charge

ion = ionform(str(g.symbol)+charge)

print(myTemplate(template_string).safe_substitute(locals()))

	
\end{pycode}


\vspace{\stretch{0.2}}

%%%%%%%%%%%%%%%%%%%%%%%%%%%%%%%%%%%%%%

\titledquestion{GVEQ8268} %%%SLO:xzenop

\begin{pycode}

template_string = r'''
Make a drawing (and brief explanation if desired) that clearly illustrates what is represented by the following symbol: \ce{^{@massnumber}@ion}

\begin{solution}
Drawing with @protons protons and @neutrons neutrons in the nucleus and @electrons electrons outside nucleus. (Like CA1 Model 1) 

\vspace{2\baselineskip}
\ifgradingnotes
Grading Note: A proficient answer must represent protons, neutrons, and electrons as particles (Not as letters or element symbol). Proton and neutron particles should be in the center (nucleus) and electrons should be outside spaced apart (ignore any shell designation). Must have correct number of protons and electrons. Minor error if number of neutrons is incorrect. 
\fi
\end{solution}

'''

g = pt.elements[randrange(1,4)]
maxabundance = 0
for i in range(len(g.isotopes)):
	if g[g.isotopes[i]].abundance > maxabundance:
		iso = i
		maxabundance = g[g.isotopes[i]].abundance
massnumber = g.isotopes[iso]
protons = g.number
neutrons = g.isotopes[iso]-protons
charge = 0
while charge == 0:
	charge = randrange(-3,protons)
electrons = protons - charge
if charge > 0:
	charge = eatOnes(charge)
	charge = '+'+charge
elif charge < 0: 
	charge = eatOnes(charge)

ion = ionform(str(g.symbol)+charge)

print(myTemplate(template_string).safe_substitute(locals()))
	
\end{pycode}

\vspace{\stretch{0.5}}

