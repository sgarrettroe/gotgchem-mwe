% Copyright (C) 2023 Tricia D. Shepherd, Sean Garrett-Roe

%% A1 Solve problems with multiple correct approaches involving average mass, numbers of atoms, or moles of an element

\titledquestion{TJHQ2204}    %%%SLO:zukurc
% Note: question based on Spencer - Chapter 1 problem 92

\begin{pycode}
template_string = r'''

\sisetup{round-mode=figures,round-precision=5}

A new scale of atomic mass was defined based on the assumption 
that the mass of @myisotope (@exactisotopeMass in our units) is exactly 
@myisotopeMass. What would be the atomic mass of @element? Explain 
whether or not your answer seems reasonable. 
\answerline[@theanswer] 


\begin{solution}
Students should expect their answer to be greater/less than the normal 
atomic mass based on how the new mass for @myisotope{} compares to the 
original atomic mass. Relative mass scale -- shifts by ratio: 
\[@myisotopeMass\,\frac{@atwomass}{@aonemass}=@theanswer\]
\end{solution}
  
'''
atomone = []
while atomone == []: 
    atomone = randrange(1,18)
    # get rid of carbon, too easy
    if atomone == 12:
        atomone = []
g = pt.elements[atomone]

# pick a random but not insane isotope
abundance = 0
while abundance==0:
    iso = choice(g.isotopes)
    exactisotopeMass = g[iso].mass
    massnumber = iso
    abundance = g[iso].abundance

mass = randint(1,20)
while mass == massnumber:
    mass = randint(1,20)
isotope = '\ce{^{' + str(massnumber) + '}' + g.symbol + '}'
atomnum = randrange(1,37)
while atomnum == atomone:
    atomnum = randrange(1,37)
g2 = pt.elements[atomnum]
soln = mass/exactisotopeMass*g2.mass

d = {'theanswer': '\SI{'+str(soln)+'}{amu}',
  'aonemass':'\SI{'+str(g.mass)+'}{amu}',
  'atwomass':'\SI{'+str(g2.mass)+'}{amu}',
  'myisotopeMass':'\SI{'+str(mass)+'}{amu}',
  'myisotope':isotope,
  'element':g2.name,
  'exactisotopeMass':'\SI{'+str(exactisotopeMass)+'}{amu}',
      }
print(myTemplate(template_string).safe_substitute(d))
\end{pycode}
\vspace{\stretch{1}}

